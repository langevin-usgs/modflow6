The SWF Model simulates one-dimension routing of surface water using the Muskingum routing approach.

\subsection{Units of Length and Time}
The SWF Model simulates surface water routing without using prescribed length and time units. Any consistent units of length and time can be used when specifying the input data for a simulation. This capability gives a certain amount of freedom to the user, but care must be exercised to avoid mixing units.  The program cannot detect the use of inconsistent units.

\subsection{Flow Budget}
A summary of all inflow (sources) and outflow (sinks) is called a flow budget.  \mf calculates a flow budget for the overall model as a check on the acceptability of the solution, and to provide a summary of the sources and sinks to the flow system.  The flow budget is printed to theSWF Model Listing File for selected time steps.

\subsection{Time Stepping}

For the present implementation of the SWF Model, all terms in the routing equation are solved explicitly.  In \mf time step lengths are controlled by the user and specified in the Temporal Discretization (TDIS) input file.  Instructions for specifying time steps are described in the TDIS section of this user guide.  

\newpage
\subsection{SWF Model Name File}
\input{swf/namefile.tex}

\newpage
\subsection{Line Network Discretization (DISL) Package}
\input{swf/disl}

\newpage
\subsection{Muskingum Routing (MMR) Package}
Input to the Muskingum Routing (MMR) Package is read from the file that has type ``MMR6'' in the Name File.  A single MMR Package is required for each SWF model. 

\vspace{5mm}
\subsubsection{Structure of Blocks}
\lstinputlisting[style=blockdefinition]{./mf6ivar/tex/swf-mmr-options.dat}
\lstinputlisting[style=blockdefinition]{./mf6ivar/tex/swf-mmr-griddata.dat}

\vspace{5mm}
\subsubsection{Explanation of Variables}
\begin{description}
% DO NOT MODIFY THIS FILE DIRECTLY.  IT IS CREATED BY mf6ivar.py 

\item \textbf{Block: OPTIONS}

\begin{description}
\item \texttt{SAVE\_FLOWS}---keyword to indicate that budget flow terms will be written to the file specified with ``BUDGET SAVE FILE'' in Output Control.

\item \texttt{PRINT\_FLOWS}---keyword to indicate that calculated flows between cells will be printed to the listing file for every stress period time step in which ``BUDGET PRINT'' is specified in Output Control. If there is no Output Control option and ``PRINT\_FLOWS'' is specified, then flow rates are printed for the last time step of each stress period.  This option can produce extremely large list files because all cell-by-cell flows are printed.  It should only be used with the MMR Package for models that have a small number of cells.

\item \texttt{OBS6}---keyword to specify that record corresponds to an observations file.

\item \texttt{FILEIN}---keyword to specify that an input filename is expected next.

\item \texttt{obs6\_filename}---name of input file to define observations for the MMR package. See the ``Observation utility'' section for instructions for preparing observation input files. Tables \ref{table:gwf-obstypetable} and \ref{table:gwt-obstypetable} lists observation type(s) supported by the MMR package.

\end{description}
\item \textbf{Block: GRIDDATA}

\begin{description}
\item \texttt{iseg\_order}---segment calculation order

\item \texttt{qoutflow0}---initial outflow from the reach at time zero

\item \texttt{k\_coef}---manning K coefficient

\item \texttt{x\_coef}---The amount of attenuation of the flow wave, called the Muskingum routing weighting factor; enter 0.0 for reservoirs, diversions, and segment(s) flowing out of the basin

\end{description}


\end{description}

\vspace{5mm}
\subsubsection{Example Input File}
\lstinputlisting[style=inputfile]{./mf6ivar/examples/swf-mmr-example.dat}



\newpage
\subsection{Muskingum-Cunge-Todini (MCT) Package}
Input to the Muskingum-Cunge-Todini (MCT) Package is read from the file that has type ``MCT6'' in the Name File.  One routing package, either a MCT or MMR Package, is required for each SWF model. 

\vspace{5mm}
\subsubsection{Structure of Blocks}
\lstinputlisting[style=blockdefinition]{./mf6ivar/tex/swf-mct-options.dat}
\lstinputlisting[style=blockdefinition]{./mf6ivar/tex/swf-mct-griddata.dat}

\vspace{5mm}
\subsubsection{Explanation of Variables}
\begin{description}
% DO NOT MODIFY THIS FILE DIRECTLY.  IT IS CREATED BY mf6ivar.py 

\item \textbf{Block: OPTIONS}

\begin{description}
\item \texttt{unitconv}---real value defining the numerator used in Manning's equation to convert between flow units.  The default value of 1.0 applies when the length dimension for flow is meters.  For English units with feet used for flow, this value should be specified as 1.49.

\item \texttt{SAVE\_FLOWS}---keyword to indicate that budget flow terms will be written to the file specified with ``BUDGET SAVE FILE'' in Output Control.

\item \texttt{PRINT\_FLOWS}---keyword to indicate that calculated flows between cells will be printed to the listing file for every stress period time step in which ``BUDGET PRINT'' is specified in Output Control. If there is no Output Control option and ``PRINT\_FLOWS'' is specified, then flow rates are printed for the last time step of each stress period.  This option can produce extremely large list files because all cell-by-cell flows are printed.  It should only be used with the MCT Package for models that have a small number of cells.

\item \texttt{OBS6}---keyword to specify that record corresponds to an observations file.

\item \texttt{FILEIN}---keyword to specify that an input filename is expected next.

\item \texttt{obs6\_filename}---name of input file to define observations for the MCT package. See the ``Observation utility'' section for instructions for preparing observation input files. Tables \ref{table:gwf-obstypetable} and \ref{table:gwt-obstypetable} lists observation type(s) supported by the MCT package.

\end{description}
\item \textbf{Block: GRIDDATA}

\begin{description}
\item \texttt{icalc\_order}---reach calculation order

\item \texttt{qoutflow0}---initial outflow from the reach at time zero

\item \texttt{width}---real value that defines the reach width. WIDTH must be greater than zero.

\item \texttt{manningsn}---mannings roughness coefficient

\item \texttt{elevation}---real value that defines the elevation of the reach.  Stream stage is calculated by adding the calculated depth to this elevation value.

\item \texttt{slope}---bottom slope of the river bed

\item \texttt{idcxs}---integer value indication the cross section identifier in the Cross Section Package that applies to the reach.

\end{description}


\end{description}

\vspace{5mm}
\subsubsection{Example Input File}
\lstinputlisting[style=inputfile]{./mf6ivar/examples/swf-mct-example.dat}



\newpage
\subsection{Cross Section (CXS) Package}
\input{swf/cxs}

%\newpage
%\subsection{Initial Conditions (IC) Package}
%\input{swf/ic}

\newpage
\subsection{Output Control (OC) Option}
\input{swf/oc}

%\newpage
%\subsection{Observation (OBS) Utility for a GWT Model}
%\input{swf/swf-obs}

\newpage
\subsection{Inflow (FLW) Package}
\input{swf/flw}


