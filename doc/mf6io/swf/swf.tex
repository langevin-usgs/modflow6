The SWF Model simulates one-dimension routing of surface water using the Muskingum routing approach.

\subsection{Units of Length and Time}
The SWF Model simulates surface water routing without using prescribed length and time units. Any consistent units of length and time can be used when specifying the input data for a simulation. This capability gives a certain amount of freedom to the user, but care must be exercised to avoid mixing units.  The program cannot detect the use of inconsistent units.

\subsection{Flow Budget}
A summary of all inflow (sources) and outflow (sinks) is called a flow budget.  \mf calculates a flow budget for the overall model as a check on the acceptability of the solution, and to provide a summary of the sources and sinks to the flow system.  The flow budget is printed to theSWF Model Listing File for selected time steps.

\subsection{Time Stepping}

For the present implementation of the SWF Model, all terms in the routing equation are solved explicitly.  In \mf time step lengths are controlled by the user and specified in the Temporal Discretization (TDIS) input file.  Instructions for specifying time steps are described in the TDIS section of this user guide.  

\newpage
\subsection{SWF Model Name File}
The SWF Model Name File specifies the options and packages that are active for a SWF model.  The Name File contains two blocks: OPTIONS  and PACKAGES. The lines in each block can be in any order.  Files listed in the PACKAGES block must exist when the program starts. 

Comment lines are indicated when the first character in a line is one of the valid comment characters.  Commented lines can be located anywhere in the file. Any text characters can follow the comment character. Comment lines have no effect on the simulation; their purpose is to allow users to provide documentation about a particular simulation. 

\vspace{5mm}
\subsubsection{Structure of Blocks}
%\lstinputlisting[style=blockdefinition]{./mf6ivar/tex/swf-nam-options.dat}
%\lstinputlisting[style=blockdefinition]{./mf6ivar/tex/swf-nam-packages.dat}

\vspace{5mm}
\subsubsection{Explanation of Variables}
%\begin{description}
%% DO NOT MODIFY THIS FILE DIRECTLY.  IT IS CREATED BY mf6ivar.py 

\item \textbf{Block: OPTIONS}

\begin{description}
\item \texttt{list}---is name of the listing file to create for this SWF model.  If not specified, then the name of the list file will be the basename of the SWF model name file and the '.lst' extension.  For example, if the SWF name file is called ``my.model.nam'' then the list file will be called ``my.model.lst''.

\item \texttt{PRINT\_INPUT}---keyword to indicate that the list of all model stress package information will be written to the listing file immediately after it is read.

\item \texttt{PRINT\_FLOWS}---keyword to indicate that the list of all model package flow rates will be printed to the listing file for every stress period time step in which ``BUDGET PRINT'' is specified in Output Control.  If there is no Output Control option and ``PRINT\_FLOWS'' is specified, then flow rates are printed for the last time step of each stress period.

\item \texttt{SAVE\_FLOWS}---keyword to indicate that all model package flow terms will be written to the file specified with ``BUDGET FILEOUT'' in Output Control.

\item \texttt{NEWTON}---keyword that activates the Newton-Raphson formulation for surface water flow between connected reaches and stress packages that support calculation of Newton-Raphson terms.

\item \texttt{UNDER\_RELAXATION}---keyword that indicates whether the surface water stage in a reach will be under-relaxed when water levels fall below the bottom of the model below any given cell. By default, Newton-Raphson UNDER\_RELAXATION is not applied.

\end{description}
\item \textbf{Block: PACKAGES}

\begin{description}
\item \texttt{ftype}---is the file type, which must be one of the following character values shown in table~\ref{table:ftype}. Ftype may be entered in any combination of uppercase and lowercase.

\item \texttt{fname}---is the name of the file containing the package input.  The path to the file should be included if the file is not located in the folder where the program was run.

\item \texttt{pname}---is the user-defined name for the package. PNAME is restricted to 16 characters.  No spaces are allowed in PNAME.  PNAME character values are read and stored by the program for stress packages only.  These names may be useful for labeling purposes when multiple stress packages of the same type are located within a single SWF Model.  If PNAME is specified for a stress package, then PNAME will be used in the flow budget table in the listing file; it will also be used for the text entry in the cell-by-cell budget file.  PNAME is case insensitive and is stored in all upper case letters.

\end{description}


%\end{description}

\begin{table}[H]
\caption{Ftype values described in this report.  The \texttt{Pname} column indicates whether or not a package name can be provided in the name file.  The capability to provide a package name also indicates that the SWF Model can have more than one package of that Ftype}
\small
\begin{center}
\begin{tabular*}{\columnwidth}{l l l}
\hline
\hline
Ftype & Input File Description & \texttt{Pname}\\
\hline
DISL6 & Line Network Discretization Input File \\
%IC6 & Initial Conditions Package \\
OC6 & Output Control Option \\
MMR6 & Muskingum Routing Package \\ 
MCT6 & Muskingum-Cunge-Todini Package \\ 
CXS6 & Cross Section Package \\ 
FLW6 & Inflow Package & * \\ 
%OBS6 & Observations Option \\
\hline 
\end{tabular*}
\label{table:ftype}
\end{center}
\normalsize
\end{table}

\vspace{5mm}
\subsubsection{Example Input File}
%\lstinputlisting[style=inputfile]{./mf6ivar/examples/swf-nam-example.dat}



\newpage
\subsection{Line Network Discretization (DISL) Package}
Input to the Line Discretization (DISL) Package is read from the file that has type ``DISL6'' in the Name File.  Only one DISL Package can be specified for a SWF Model.

The DISL VERTICES and CELL2D blocks are not required.  In general, it is recommended to include the VERTICES and CELL2D blocks so that external programs can plot the line network. 


\vspace{5mm}
\subsubsection{Structure of Blocks}
\lstinputlisting[style=blockdefinition]{./mf6ivar/tex/swf-disl-options.dat}
\lstinputlisting[style=blockdefinition]{./mf6ivar/tex/swf-disl-dimensions.dat}
\lstinputlisting[style=blockdefinition]{./mf6ivar/tex/swf-disl-griddata.dat}
\lstinputlisting[style=blockdefinition]{./mf6ivar/tex/swf-disl-vertices.dat}
\lstinputlisting[style=blockdefinition]{./mf6ivar/tex/swf-disl-cell2d.dat}

\vspace{5mm}
\subsubsection{Explanation of Variables}
\begin{description}
% DO NOT MODIFY THIS FILE DIRECTLY.  IT IS CREATED BY mf6ivar.py 

\item \textbf{Block: OPTIONS}

\begin{description}
\item \texttt{length\_units}---is the length units used for this model.  Values can be ``FEET'', ``METERS'', or ``CENTIMETERS''.  If not specified, the default is ``UNKNOWN''.

\item \texttt{length\_convert}---conversion factor that is used in converting constants used by MODFLOW into model length units.  All constants are by default in units of ``meters''. Constants that use length\_conversion include GRAVITY and STORAGE COEFFICIENT values.

\item \texttt{time\_convert}---conversion factor that is used in converting constants used by MODFLOW into model time units.  All constants are by default in units of ``seconds''. TIME\_CONVERSION is used to convert the GRAVITY constant from seconds to the model's units.

\item \texttt{NOGRB}---keyword to deactivate writing of the binary grid file.

\item \texttt{xorigin}---x-position of the origin used for model grid vertices.  This value should be provided in a real-world coordinate system.  A default value of zero is assigned if not specified.  The value for XORIGIN does not affect the model simulation, but it is written to the binary grid file so that postprocessors can locate the grid in space.

\item \texttt{yorigin}---y-position of the origin used for model grid vertices.  This value should be provided in a real-world coordinate system.  If not specified, then a default value equal to zero is used.  The value for YORIGIN does not affect the model simulation, but it is written to the binary grid file so that postprocessors can locate the grid in space.

\item \texttt{angrot}---counter-clockwise rotation angle (in degrees) of the model grid coordinate system relative to a real-world coordinate system.  If not specified, then a default value of 0.0 is assigned.  The value for ANGROT does not affect the model simulation, but it is written to the binary grid file so that postprocessors can locate the grid in space.

\end{description}
\item \textbf{Block: DIMENSIONS}

\begin{description}
\item \texttt{nodes}---is the number of linear cells.

\item \texttt{nvert}---is the total number of (x, y, z) vertex pairs used to characterize the model grid.

\end{description}
\item \textbf{Block: GRIDDATA}

\begin{description}
\item \texttt{reach\_length}---length for each reach

\item \texttt{reach\_bottom}---bottom elevation of surface water channel

\item \texttt{toreach}---index of the downstream reach.  Flow from this reach is passed into the dowstream reach.  For reaches that do not flow to another reach enter a 0 to toreach.

\item \texttt{idomain}---is an optional array that characterizes the existence status of a cell.  If the IDOMAIN array is not specified, then all model cells exist within the solution.  If the IDOMAIN value for a cell is 0, the cell does not exist in the simulation.  Input and output values will be read and written for the cell, but internal to the program, the cell is excluded from the solution.  If the IDOMAIN value for a cell is 1, the cell exists in the simulation.

\end{description}
\item \textbf{Block: VERTICES}

\begin{description}
\item \texttt{iv}---is the vertex number.  Records in the VERTICES block must be listed in consecutive order from 1 to NVERT.

\item \texttt{xv}---is the x-coordinate for the vertex.

\item \texttt{yv}---is the y-coordinate for the vertex.

\item \texttt{zv}---is the z-coordinate for the vertex.

\end{description}
\item \textbf{Block: CELL2D}

\begin{description}
\item \texttt{icell2d}---is the cell2d number.  Records in the cell2d block must be listed in consecutive order from the first to the last.

\item \texttt{fdc}---is the fractional distance to the cell center. FDC is relative to the first vertex in the ICVERT array. In most cases FDC should be 0.5, which would place the center of the line segment that defines the cell. If the value of FDC is 1, the cell center would located at the last vertex. FDC values of 0 and 1 can be used to place the node at either end of the cell which can be useful for cells with boundary conditions.

\item \texttt{ncvert}---is the number of vertices required to define the cell.  There may be a different number of vertices for each cell.

\item \texttt{icvert}---is an array of integer values containing vertex numbers (in the VERTICES block) used to define the cell.  Vertices must be listed in the order that defines the line representing the cell.  Cells that are connected must share vertices. The bottom elevation of the cell is calculated using the ZV of the first and last vertex point and FDC.

\end{description}


\end{description}

\vspace{5mm}
\subsubsection{Example Input File}
\lstinputlisting[style=inputfile]{./mf6ivar/examples/swf-disl-example.dat}


\newpage
\subsection{Muskingum Routing (MMR) Package}
Input to the Muskingum Routing (MMR) Package is read from the file that has type ``MMR6'' in the Name File.  A single MMR Package is required for each SWF model. 

\vspace{5mm}
\subsubsection{Structure of Blocks}
\lstinputlisting[style=blockdefinition]{./mf6ivar/tex/swf-mmr-options.dat}
\lstinputlisting[style=blockdefinition]{./mf6ivar/tex/swf-mmr-griddata.dat}

\vspace{5mm}
\subsubsection{Explanation of Variables}
\begin{description}
% DO NOT MODIFY THIS FILE DIRECTLY.  IT IS CREATED BY mf6ivar.py 

\item \textbf{Block: OPTIONS}

\begin{description}
\item \texttt{SAVE\_FLOWS}---keyword to indicate that budget flow terms will be written to the file specified with ``BUDGET SAVE FILE'' in Output Control.

\item \texttt{PRINT\_FLOWS}---keyword to indicate that calculated flows between cells will be printed to the listing file for every stress period time step in which ``BUDGET PRINT'' is specified in Output Control. If there is no Output Control option and ``PRINT\_FLOWS'' is specified, then flow rates are printed for the last time step of each stress period.  This option can produce extremely large list files because all cell-by-cell flows are printed.  It should only be used with the MMR Package for models that have a small number of cells.

\item \texttt{OBS6}---keyword to specify that record corresponds to an observations file.

\item \texttt{FILEIN}---keyword to specify that an input filename is expected next.

\item \texttt{obs6\_filename}---name of input file to define observations for the MMR package. See the ``Observation utility'' section for instructions for preparing observation input files. Tables \ref{table:gwf-obstypetable} and \ref{table:gwt-obstypetable} lists observation type(s) supported by the MMR package.

\end{description}
\item \textbf{Block: GRIDDATA}

\begin{description}
\item \texttt{iseg\_order}---segment calculation order

\item \texttt{qoutflow0}---initial outflow from the reach at time zero

\item \texttt{k\_coef}---manning K coefficient

\item \texttt{x\_coef}---The amount of attenuation of the flow wave, called the Muskingum routing weighting factor; enter 0.0 for reservoirs, diversions, and segment(s) flowing out of the basin

\end{description}


\end{description}

\vspace{5mm}
\subsubsection{Example Input File}
\lstinputlisting[style=inputfile]{./mf6ivar/examples/swf-mmr-example.dat}



\newpage
\subsection{Muskingum-Cunge-Todini (MCT) Package}
Input to the Muskingum-Cunge-Todini (MCT) Package is read from the file that has type ``MCT6'' in the Name File.  One routing package, either a MCT or MMR Package, is required for each SWF model. 

\vspace{5mm}
\subsubsection{Structure of Blocks}
\lstinputlisting[style=blockdefinition]{./mf6ivar/tex/swf-mct-options.dat}
\lstinputlisting[style=blockdefinition]{./mf6ivar/tex/swf-mct-griddata.dat}

\vspace{5mm}
\subsubsection{Explanation of Variables}
\begin{description}
% DO NOT MODIFY THIS FILE DIRECTLY.  IT IS CREATED BY mf6ivar.py 

\item \textbf{Block: OPTIONS}

\begin{description}
\item \texttt{unitconv}---real value defining the numerator used in Manning's equation to convert between flow units.  The default value of 1.0 applies when the length dimension for flow is meters.  For English units with feet used for flow, this value should be specified as 1.49.

\item \texttt{SAVE\_FLOWS}---keyword to indicate that budget flow terms will be written to the file specified with ``BUDGET SAVE FILE'' in Output Control.

\item \texttt{PRINT\_FLOWS}---keyword to indicate that calculated flows between cells will be printed to the listing file for every stress period time step in which ``BUDGET PRINT'' is specified in Output Control. If there is no Output Control option and ``PRINT\_FLOWS'' is specified, then flow rates are printed for the last time step of each stress period.  This option can produce extremely large list files because all cell-by-cell flows are printed.  It should only be used with the MCT Package for models that have a small number of cells.

\item \texttt{OBS6}---keyword to specify that record corresponds to an observations file.

\item \texttt{FILEIN}---keyword to specify that an input filename is expected next.

\item \texttt{obs6\_filename}---name of input file to define observations for the MCT package. See the ``Observation utility'' section for instructions for preparing observation input files. Tables \ref{table:gwf-obstypetable} and \ref{table:gwt-obstypetable} lists observation type(s) supported by the MCT package.

\end{description}
\item \textbf{Block: GRIDDATA}

\begin{description}
\item \texttt{icalc\_order}---reach calculation order

\item \texttt{qoutflow0}---initial outflow from the reach at time zero

\item \texttt{width}---real value that defines the reach width. WIDTH must be greater than zero.

\item \texttt{manningsn}---mannings roughness coefficient

\item \texttt{elevation}---real value that defines the elevation of the reach.  Stream stage is calculated by adding the calculated depth to this elevation value.

\item \texttt{slope}---bottom slope of the river bed

\item \texttt{idcxs}---integer value indication the cross section identifier in the Cross Section Package that applies to the reach.

\end{description}


\end{description}

\vspace{5mm}
\subsubsection{Example Input File}
\lstinputlisting[style=inputfile]{./mf6ivar/examples/swf-mct-example.dat}



\newpage
\subsection{Cross Section (CXS) Package}
Input to the Cross Section (CXS) Package is read from the file that has type ``CXS6'' in the Name File.  One CXS Package can be specified for each SWF model. 

\vspace{5mm}
\subsubsection{Structure of Blocks}
\lstinputlisting[style=blockdefinition]{./mf6ivar/tex/swf-cxs-options.dat}
\lstinputlisting[style=blockdefinition]{./mf6ivar/tex/swf-cxs-dimensions.dat}
\lstinputlisting[style=blockdefinition]{./mf6ivar/tex/swf-cxs-packagedata.dat}
\lstinputlisting[style=blockdefinition]{./mf6ivar/tex/swf-cxs-crosssectiondata.dat}

\vspace{5mm}
\subsubsection{Explanation of Variables}
\begin{description}
% DO NOT MODIFY THIS FILE DIRECTLY.  IT IS CREATED BY mf6ivar.py 

\item \textbf{Block: OPTIONS}

\begin{description}
\item \texttt{PRINT\_INPUT}---keyword to indicate that the list of stream reach information will be written to the listing file immediately after it is read.

\end{description}
\item \textbf{Block: DIMENSIONS}

\begin{description}
\item \texttt{nsections}---integer value specifying the number of cross sections that will be defined.  There must be NSECTIONS entries in the PACKAGEDATA block.

\item \texttt{npoints}---integer value specifying the total number of cross-section points defined for all reaches.  There must be NPOINTS entries in the CROSSSECTIONDATA block.

\end{description}
\item \textbf{Block: PACKAGEDATA}

\begin{description}
\item \texttt{idcxs}---integer value that defines the cross section number associated with the specified PACKAGEDATA data on the line. IDCXS must be greater than zero and less than or equal to NSECTIONS. Information must be specified for every section or the program will terminate with an error.  The program will also terminate with an error if information for a section is specified more than once.

\item \texttt{nxspoints}---integer value that defines the number of points used to define the define the shape of a section.  NXSPOINTS must be greater than zero or the program will terminate with an error.  NXSPOINTS defines the number of points that must be entered for the reach in the CROSSSECTIONDATA block.  The sum of NXSPOINTS for all sections must equal the NPOINTS dimension.

\end{description}
\item \textbf{Block: CROSSSECTIONDATA}

\begin{description}
\item \texttt{xfraction}---real value that defines the station (x) data for the cross-section as a fraction of the width (WIDTH) of the reach. XFRACTION must be greater than or equal to zero but can be greater than one. XFRACTION values can be used to decrease or increase the width of a reach from the specified reach width (WIDTH).

\item \texttt{height}---real value that is the height relative to the top of the lowest elevation of the streambed (ELEVATION) and corresponding to the station data on the same line. HEIGHT must be greater than or equal to zero and at least one cross-section height must be equal to zero.

\item \texttt{manfraction}---real value that defines the Manning's roughness coefficient data for the cross-section as a fraction of the Manning's roughness coefficient for the reach (MANNINGSN) and corresponding to the station data on the same line. MANFRACTION must be greater than zero. MANFRACTION is applied from the XFRACTION value on the same line to the XFRACTION value on the next line.

\end{description}


\end{description}

\vspace{5mm}
\subsubsection{Example Input File}
\lstinputlisting[style=inputfile]{./mf6ivar/examples/swf-cxs-example.dat}



%\newpage
%\subsection{Initial Conditions (IC) Package}
%\input{swf/ic}

\newpage
\subsection{Output Control (OC) Option}
Input to the Output Control Option of the Surface Water Flow Model is read from the file that is specified as type ``OC6'' in the Name File. If no ``OC6'' file is specified, default output control is used. The Output Control Option determines how and when outflows are printed to the listing file and/or written to a separate binary output file.  Under the default, outflow and overall flow budget are written to the Listing File at the end of every stress period. The default printout format for concentrations is 10G11.4.  The outflow and overall flow budget are also written to the list file if the simulation terminates prematurely due to failed convergence.

\vspace{5mm}
\subsubsection{Structure of Blocks}
\vspace{5mm}

\noindent \textit{FOR EACH SIMULATION}
\lstinputlisting[style=blockdefinition]{./mf6ivar/tex/swf-oc-options.dat}
\vspace{5mm}
\noindent \textit{FOR ANY STRESS PERIOD}
\lstinputlisting[style=blockdefinition]{./mf6ivar/tex/swf-oc-period.dat}

\vspace{5mm}
\subsubsection{Explanation of Variables}
\begin{description}
% DO NOT MODIFY THIS FILE DIRECTLY.  IT IS CREATED BY mf6ivar.py 

\item \textbf{Block: OPTIONS}

\begin{description}
\item \texttt{BUDGET}---keyword to specify that record corresponds to the budget.

\item \texttt{FILEOUT}---keyword to specify that an output filename is expected next.

\item \texttt{budgetfile}---name of the output file to write budget information.

\item \texttt{BUDGETCSV}---keyword to specify that record corresponds to the budget CSV.

\item \texttt{budgetcsvfile}---name of the comma-separated value (CSV) output file to write budget summary information.  A budget summary record will be written to this file for each time step of the simulation.

\item \texttt{QOUTFLOW}---keyword to specify that record corresponds to qoutflow.

\item \texttt{qoutflowfile}---name of the output file to write conc information.

\item \texttt{STAGE}---keyword to specify that record corresponds to stage.

\item \texttt{stagefile}---name of the output file to write stage information.

\item \texttt{PRINT\_FORMAT}---keyword to specify format for printing to the listing file.

\item \texttt{columns}---number of columns for writing data.

\item \texttt{width}---width for writing each number.

\item \texttt{digits}---number of digits to use for writing a number.

\item \texttt{format}---write format can be EXPONENTIAL, FIXED, GENERAL, or SCIENTIFIC.

\end{description}
\item \textbf{Block: PERIOD}

\begin{description}
\item \texttt{iper}---integer value specifying the starting stress period number for which the data specified in the PERIOD block apply.  IPER must be less than or equal to NPER in the TDIS Package and greater than zero.  The IPER value assigned to a stress period block must be greater than the IPER value assigned for the previous PERIOD block.  The information specified in the PERIOD block will continue to apply for all subsequent stress periods, unless the program encounters another PERIOD block.

\item \texttt{SAVE}---keyword to indicate that information will be saved this stress period.

\item \texttt{PRINT}---keyword to indicate that information will be printed this stress period.

\item \texttt{rtype}---type of information to save or print.  Can be BUDGET.

\item \texttt{ocsetting}---specifies the steps for which the data will be saved.

\begin{lstlisting}[style=blockdefinition]
ALL
FIRST
LAST
FREQUENCY <frequency>
STEPS <steps(<nstp)>
\end{lstlisting}

\item \texttt{ALL}---keyword to indicate save for all time steps in period.

\item \texttt{FIRST}---keyword to indicate save for first step in period. This keyword may be used in conjunction with other keywords to print or save results for multiple time steps.

\item \texttt{LAST}---keyword to indicate save for last step in period. This keyword may be used in conjunction with other keywords to print or save results for multiple time steps.

\item \texttt{frequency}---save at the specified time step frequency. This keyword may be used in conjunction with other keywords to print or save results for multiple time steps.

\item \texttt{steps}---save for each step specified in STEPS. This keyword may be used in conjunction with other keywords to print or save results for multiple time steps.

\end{description}


\end{description}

\vspace{5mm}
\subsubsection{Example Input File}
\lstinputlisting[style=inputfile]{./mf6ivar/examples/swf-oc-example.dat}


%\newpage
%\subsection{Observation (OBS) Utility for a GWT Model}
%\input{swf/swf-obs}

\newpage
\subsection{Inflow (FLW) Package}
Input to the Inflow (FLW) Package is read from the file that has type ``FLW6'' in the Name File.  Any number of FLW Packages can be specified for a single surface water flow model.

\vspace{5mm}
\subsubsection{Structure of Blocks}
\vspace{5mm}

\noindent \textit{FOR EACH SIMULATION}
\lstinputlisting[style=blockdefinition]{./mf6ivar/tex/swf-flw-options.dat}
\lstinputlisting[style=blockdefinition]{./mf6ivar/tex/swf-flw-dimensions.dat}
\vspace{5mm}
\noindent \textit{FOR ANY STRESS PERIOD}
\lstinputlisting[style=blockdefinition]{./mf6ivar/tex/swf-flw-period.dat}
\packageperioddescription

\vspace{5mm}
\subsubsection{Explanation of Variables}
\begin{description}
% DO NOT MODIFY THIS FILE DIRECTLY.  IT IS CREATED BY mf6ivar.py 

\item \textbf{Block: OPTIONS}

\begin{description}
\item \texttt{auxiliary}---defines an array of one or more auxiliary variable names.  There is no limit on the number of auxiliary variables that can be provided on this line; however, lists of information provided in subsequent blocks must have a column of data for each auxiliary variable name defined here.   The number of auxiliary variables detected on this line determines the value for naux.  Comments cannot be provided anywhere on this line as they will be interpreted as auxiliary variable names.  Auxiliary variables may not be used by the package, but they will be available for use by other parts of the program.  The program will terminate with an error if auxiliary variables are specified on more than one line in the options block.

\item \texttt{auxmultname}---name of auxiliary variable to be used as multiplier of flow rate.

\item \texttt{BOUNDNAMES}---keyword to indicate that boundary names may be provided with the list of inflow cells.

\item \texttt{PRINT\_INPUT}---keyword to indicate that the list of inflow information will be written to the listing file immediately after it is read.

\item \texttt{PRINT\_FLOWS}---keyword to indicate that the list of inflow flow rates will be printed to the listing file for every stress period time step in which ``BUDGET PRINT'' is specified in Output Control.  If there is no Output Control option and ``PRINT\_FLOWS'' is specified, then flow rates are printed for the last time step of each stress period.

\item \texttt{SAVE\_FLOWS}---keyword to indicate that inflow flow terms will be written to the file specified with ``BUDGET FILEOUT'' in Output Control.

\item \texttt{TS6}---keyword to specify that record corresponds to a time-series file.

\item \texttt{FILEIN}---keyword to specify that an input filename is expected next.

\item \texttt{ts6\_filename}---defines a time-series file defining time series that can be used to assign time-varying values. See the ``Time-Variable Input'' section for instructions on using the time-series capability.

\item \texttt{OBS6}---keyword to specify that record corresponds to an observations file.

\item \texttt{obs6\_filename}---name of input file to define observations for the inflow package. See the ``Observation utility'' section for instructions for preparing observation input files. Tables \ref{table:gwf-obstypetable} and \ref{table:gwt-obstypetable} lists observation type(s) supported by the inflow package.

\end{description}
\item \textbf{Block: DIMENSIONS}

\begin{description}
\item \texttt{maxbound}---integer value specifying the maximum number of inflow cells that will be specified for use during any stress period.

\end{description}
\item \textbf{Block: PERIOD}

\begin{description}
\item \texttt{iper}---integer value specifying the starting stress period number for which the data specified in the PERIOD block apply.  IPER must be less than or equal to NPER in the TDIS Package and greater than zero.  The IPER value assigned to a stress period block must be greater than the IPER value assigned for the previous PERIOD block.  The information specified in the PERIOD block will continue to apply for all subsequent stress periods, unless the program encounters another PERIOD block.

\item \texttt{cellid}---is the cell identifier, and depends on the type of grid that is used for the simulation.  For a structured grid that uses the DIS input file, CELLID is the layer, row, and column.   For a grid that uses the DISV input file, CELLID is the layer and CELL2D number.  If the model uses the unstructured discretization (DISU) input file, CELLID is the node number for the cell.

\item \textcolor{blue}{\texttt{q}---is the volumetric inflow rate. A positive value indicates inflow to the stream.  Negative values are not allows. If the Options block includes a TIMESERIESFILE entry (see the ``Time-Variable Input'' section), values can be obtained from a time series by entering the time-series name in place of a numeric value.}

\item \textcolor{blue}{\texttt{aux}---represents the values of the auxiliary variables for each inflow. The values of auxiliary variables must be present for each inflow. The values must be specified in the order of the auxiliary variables specified in the OPTIONS block.  If the package supports time series and the Options block includes a TIMESERIESFILE entry (see the ``Time-Variable Input'' section), values can be obtained from a time series by entering the time-series name in place of a numeric value.}

\item \texttt{boundname}---name of the inflow cell.  BOUNDNAME is an ASCII character variable that can contain as many as 40 characters.  If BOUNDNAME contains spaces in it, then the entire name must be enclosed within single quotes.

\end{description}


\end{description}

\vspace{5mm}
\subsubsection{Example Input File}
\lstinputlisting[style=inputfile]{./mf6ivar/examples/swf-flw-example.dat}

%\vspace{5mm}
%\subsubsection{Available observation types}
%Well Package observations include the simulated well rates (\texttt{wel}), the well discharge that is available for the MVR package (\texttt{to-mvr}), and the reduction in the specified \texttt{q} when the \texttt{AUTO\_FLOW\_REDUCE} option is enabled. The data required for each WEL Package observation type is defined in table~\ref{table:gwf-welobstype}. The sum of \texttt{wel} and \texttt{to-mvr} is equal to the simulated well discharge rate, which may be less than the specified \texttt{q} if the \texttt{AUTO\_FLOW\_REDUCE} option is enabled. The \texttt{DNODATA} value is returned if the \texttt{wel-reduction} observation is specified but the \texttt{AUTO\_FLOW\_REDUCE} option is not enabled. Negative and positive values for an observation represent a loss from and gain to the GWF model, respectively.

%\begin{longtable}{p{2cm} p{2.75cm} p{2cm} p{1.25cm} p{7cm}}
%\caption{Available WEL Package observation types} \tabularnewline

%\hline
%\hline
%\textbf{Stress Package} & \textbf{Observation type} & \textbf{ID} & \textbf{ID2} & \textbf{Description} \\
%\hline
%\endhead

%\hline
%\endfoot

%\input{../Common/gwf-welobs.tex}
%\label{table:gwf-welobstype}
%\end{longtable}

%\vspace{5mm}
%\subsubsection{Example Observation Input File}
%\lstinputlisting[style=inputfile]{./mf6ivar/examples/gwf-wel-example-obs.dat}



