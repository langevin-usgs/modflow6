The SWF Model Name File specifies the options and packages that are active for a SWF model.  The Name File contains two blocks: OPTIONS  and PACKAGES. The lines in each block can be in any order.  Files listed in the PACKAGES block must exist when the program starts. 

Comment lines are indicated when the first character in a line is one of the valid comment characters.  Commented lines can be located anywhere in the file. Any text characters can follow the comment character. Comment lines have no effect on the simulation; their purpose is to allow users to provide documentation about a particular simulation. 

\vspace{5mm}
\subsubsection{Structure of Blocks}
%\lstinputlisting[style=blockdefinition]{./mf6ivar/tex/swf-nam-options.dat}
%\lstinputlisting[style=blockdefinition]{./mf6ivar/tex/swf-nam-packages.dat}

\vspace{5mm}
\subsubsection{Explanation of Variables}
%\begin{description}
%% DO NOT MODIFY THIS FILE DIRECTLY.  IT IS CREATED BY mf6ivar.py 

\item \textbf{Block: OPTIONS}

\begin{description}
\item \texttt{list}---is name of the listing file to create for this SWF model.  If not specified, then the name of the list file will be the basename of the SWF model name file and the '.lst' extension.  For example, if the SWF name file is called ``my.model.nam'' then the list file will be called ``my.model.lst''.

\item \texttt{PRINT\_INPUT}---keyword to indicate that the list of all model stress package information will be written to the listing file immediately after it is read.

\item \texttt{PRINT\_FLOWS}---keyword to indicate that the list of all model package flow rates will be printed to the listing file for every stress period time step in which ``BUDGET PRINT'' is specified in Output Control.  If there is no Output Control option and ``PRINT\_FLOWS'' is specified, then flow rates are printed for the last time step of each stress period.

\item \texttt{SAVE\_FLOWS}---keyword to indicate that all model package flow terms will be written to the file specified with ``BUDGET FILEOUT'' in Output Control.

\item \texttt{NEWTON}---keyword that activates the Newton-Raphson formulation for surface water flow between connected reaches and stress packages that support calculation of Newton-Raphson terms.

\item \texttt{UNDER\_RELAXATION}---keyword that indicates whether the surface water stage in a reach will be under-relaxed when water levels fall below the bottom of the model below any given cell. By default, Newton-Raphson UNDER\_RELAXATION is not applied.

\end{description}
\item \textbf{Block: PACKAGES}

\begin{description}
\item \texttt{ftype}---is the file type, which must be one of the following character values shown in table~\ref{table:ftype}. Ftype may be entered in any combination of uppercase and lowercase.

\item \texttt{fname}---is the name of the file containing the package input.  The path to the file should be included if the file is not located in the folder where the program was run.

\item \texttt{pname}---is the user-defined name for the package. PNAME is restricted to 16 characters.  No spaces are allowed in PNAME.  PNAME character values are read and stored by the program for stress packages only.  These names may be useful for labeling purposes when multiple stress packages of the same type are located within a single SWF Model.  If PNAME is specified for a stress package, then PNAME will be used in the flow budget table in the listing file; it will also be used for the text entry in the cell-by-cell budget file.  PNAME is case insensitive and is stored in all upper case letters.

\end{description}


%\end{description}

\begin{table}[H]
\caption{Ftype values described in this report.  The \texttt{Pname} column indicates whether or not a package name can be provided in the name file.  The capability to provide a package name also indicates that the SWF Model can have more than one package of that Ftype}
\small
\begin{center}
\begin{tabular*}{\columnwidth}{l l l}
\hline
\hline
Ftype & Input File Description & \texttt{Pname}\\
\hline
DISL6 & Line Network Discretization Input File \\
%IC6 & Initial Conditions Package \\
OC6 & Output Control Option \\
MMR6 & Muskingum Routing Package \\ 
MCT6 & Muskingum-Cunge-Todini Package \\ 
CXS6 & Cross Section Package \\ 
FLW6 & Inflow Package & * \\ 
%OBS6 & Observations Option \\
\hline 
\end{tabular*}
\label{table:ftype}
\end{center}
\normalsize
\end{table}

\vspace{5mm}
\subsubsection{Example Input File}
%\lstinputlisting[style=inputfile]{./mf6ivar/examples/swf-nam-example.dat}

