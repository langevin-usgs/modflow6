% DO NOT MODIFY THIS FILE DIRECTLY.  IT IS CREATED BY mf6ivar.py 

\item \textbf{Block: OPTIONS}

\begin{description}
\item \texttt{PRINT\_INPUT}---keyword to indicate that the list of stream reach information will be written to the listing file immediately after it is read.

\end{description}
\item \textbf{Block: DIMENSIONS}

\begin{description}
\item \texttt{nsections}---integer value specifying the number of cross sections that will be defined.  There must be NSECTIONS entries in the PACKAGEDATA block.

\item \texttt{npoints}---integer value specifying the total number of cross-section points defined for all reaches.  There must be NPOINTS entries in the CROSSSECTIONDATA block.

\end{description}
\item \textbf{Block: PACKAGEDATA}

\begin{description}
\item \texttt{idcxs}---integer value that defines the cross section number associated with the specified PACKAGEDATA data on the line. IDCXS must be greater than zero and less than or equal to NSECTIONS. Information must be specified for every section or the program will terminate with an error.  The program will also terminate with an error if information for a section is specified more than once.

\item \texttt{nxspoints}---integer value that defines the number of points used to define the define the shape of a section.  NXSPOINTS must be greater than zero or the program will terminate with an error.  NXSPOINTS defines the number of points that must be entered for the reach in the CROSSSECTIONDATA block.  The sum of NXSPOINTS for all sections must equal the NPOINTS dimension.

\end{description}
\item \textbf{Block: CROSSSECTIONDATA}

\begin{description}
\item \texttt{xfraction}---real value that defines the station (x) data for the cross-section as a fraction of the width (WIDTH) of the reach. XFRACTION must be greater than or equal to zero but can be greater than one. XFRACTION values can be used to decrease or increase the width of a reach from the specified reach width (WIDTH).

\item \texttt{height}---real value that is the height relative to the top of the lowest elevation of the streambed (ELEVATION) and corresponding to the station data on the same line. HEIGHT must be greater than or equal to zero and at least one cross-section height must be equal to zero.

\item \texttt{manfraction}---real value that defines the Manning's roughness coefficient data for the cross-section as a fraction of the Manning's roughness coefficient for the reach (MANNINGSN) and corresponding to the station data on the same line. MANFRACTION must be greater than zero. MANFRACTION is applied from the XFRACTION value on the same line to the XFRACTION value on the next line.

\end{description}

