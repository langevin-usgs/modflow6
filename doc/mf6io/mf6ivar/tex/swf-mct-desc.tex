% DO NOT MODIFY THIS FILE DIRECTLY.  IT IS CREATED BY mf6ivar.py 

\item \textbf{Block: OPTIONS}

\begin{description}
\item \texttt{unitconv}---real value defining the numerator used in Manning's equation to convert between flow units.  The default value of 1.0 applies when the length dimension for flow is meters.  For English units with feet used for flow, this value should be specified as 1.49.

\item \texttt{SAVE\_FLOWS}---keyword to indicate that budget flow terms will be written to the file specified with ``BUDGET SAVE FILE'' in Output Control.

\item \texttt{PRINT\_FLOWS}---keyword to indicate that calculated flows between cells will be printed to the listing file for every stress period time step in which ``BUDGET PRINT'' is specified in Output Control. If there is no Output Control option and ``PRINT\_FLOWS'' is specified, then flow rates are printed for the last time step of each stress period.  This option can produce extremely large list files because all cell-by-cell flows are printed.  It should only be used with the MCT Package for models that have a small number of cells.

\item \texttt{OBS6}---keyword to specify that record corresponds to an observations file.

\item \texttt{FILEIN}---keyword to specify that an input filename is expected next.

\item \texttt{obs6\_filename}---name of input file to define observations for the MCT package. See the ``Observation utility'' section for instructions for preparing observation input files. Tables \ref{table:gwf-obstypetable} and \ref{table:gwt-obstypetable} lists observation type(s) supported by the MCT package.

\end{description}
\item \textbf{Block: GRIDDATA}

\begin{description}
\item \texttt{icalc\_order}---reach calculation order

\item \texttt{qoutflow0}---initial outflow from the reach at time zero

\item \texttt{width}---real value that defines the reach width. WIDTH must be greater than zero.

\item \texttt{manningsn}---mannings roughness coefficient

\item \texttt{elevation}---real value that defines the elevation of the reach.  Stream stage is calculated by adding the calculated depth to this elevation value.

\item \texttt{slope}---bottom slope of the river bed

\item \texttt{idcxs}---integer value indication the cross section identifier in the Cross Section Package that applies to the reach.

\end{description}

