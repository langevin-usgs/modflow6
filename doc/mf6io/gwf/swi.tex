Input to the Sea Water Intrusion (SWI) Package is read from the file that has type ``SWI6'' in the Name File.  If the SWI Package is included for a model, then the model will calculate the position of the freshwater interface in each cell and calculate flow accordingly.

For some applications, it may be necessary to specify the saltwater head as an array that changes with time.  As described in the options block for TVA6, the  \hyperref[sec:tva]{Time Variable Array (TVA) Utility} can be used to specify the saltwater head .

\vspace{5mm}
\subsubsection{Structure of Blocks}

\vspace{5mm}
\noindent \textit{FOR EACH SIMULATION}
\lstinputlisting[style=blockdefinition]{./mf6ivar/tex/gwf-swi-options.dat}
\lstinputlisting[style=blockdefinition]{./mf6ivar/tex/gwf-swi-griddata.dat}
%\vspace{5mm}
%\noindent \textit{FOR ANY STRESS PERIOD}
%\lstinputlisting[style=blockdefinition]{./mf6ivar/tex/gwf-buy-period.dat}

\vspace{5mm}
\subsubsection{Explanation of Variables}
\begin{description}
% DO NOT MODIFY THIS FILE DIRECTLY.  IT IS CREATED BY mf6ivar.py 

\item \textbf{Block: OPTIONS}

\begin{description}
\item \texttt{SALTWATER}---keyword to indicate that this model is a saltwater model

\item \texttt{ZETA}---keyword to specify that record corresponds to zeta.

\item \texttt{FILEOUT}---keyword to specify that an output filename is expected next.

\item \texttt{zetafile}---name of the binary output file to write zeta information.

\end{description}
\item \textbf{Block: GRIDDATA}

\begin{description}
\item \texttt{zetastrt}---is the initial (starting) zeta surface---that is, interface elevation at the beginning of the GWF Model simulation.

\end{description}


\end{description}

\vspace{5mm}
\subsubsection{Example Input File}
\lstinputlisting[style=inputfile]{./mf6ivar/examples/gwf-swi-example.dat}

