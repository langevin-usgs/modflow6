This section describes the data files for a \mf Groundwater Flow (GWF) Model.  A GWF Model is added to the simulation by including a GWF entry in the MODELS block of the simulation name file.

There are three types of spatial discretization approaches that can be used with the GWF Model.  Input for a GWF Model may be entered in a structured form, like for previous MODFLOW versions, in that users specify cells using their layer, row, and column indices.  Users may also work with a layered grid in which cells are defined using vertices.  In this case, users specify cells using the layer number and the cell number.  Lastly, GWF Models may be entered as fully unstructured models, in which cells are specified using only their cell number.  Once a spatial discretization approach has been selected, then all input with cell indices must be entered accordingly.

The GWF Model is designed to permit input to be gathered, as it is needed, from many different files.  Likewise, results from the model calculations can be written to a number of output files. The GWF Model Listing File is a key file to which the GWF model output is written.  As \mf runs, information about the GWF Model is written to the GWF Model Listing File, including much of the input data (as a record of the simulation) and calculated results.  Details about the files used by each package are provided in this section on the GWF Model Instructions.

\mf is further designed to allow the user to control the amount, type, and frequency of information to be output. Much of the output will be written to the Simulation and GWF Model Listing Files, but some model output can be written to other files.  The Listing Files can become very large for common models.  Text editors are useful for examining the Listing File. The GWF Model Listing File includes a summary of the input data read for all packages.  In addition, the GWF Model Listing File optionally contains calculated head controlled by time step, and the overall volumetric budget controlled by time step. The Listing Files also contain information about solver convergence and error messages.  Output to other files can include head and cell-by-cell flow terms for use in calculations external to the model or in user-supplied applications such as plotting programs.

The GWF Model reads a file called the Name File, which specifies most of the files that will be used in a simulation. Several files are always required whereas other files are optional depending on the simulation. The Output Control Package receives instructions from the user to control the amount and frequency of output.  Details about the Name File and the Output Control Package are described in this section.

\subsection{Information for Existing MODFLOW Users}
\input{gwf/info_existing_users.tex}

\subsection{Units of Length and Time}
The GWF Model formulates the groundwater flow equation without using prescribed length and time units. Any consistent units of length and time can be used when specifying the input data for a simulation. This capability gives a certain amount of freedom to the user, but care must be exercised to avoid mixing units.  The program cannot detect the use of inconsistent units.  For example, if hydraulic conductivity is entered in units of feet per day and pumpage as cubic meters per second, the program will run, but the results will be meaningless. Other processes generally are expected to work with consistent length and time units; however, other processes could conceivably place restrictions on which units are supported.

The user can set flags that specify the length and time units (see the input instructions for the Timing Module and Spatial Discretization Files), which may be useful in various parts of MODFLOW.  For example, the program will label the table of simulation time with time units if the time units are specified by the optional TIME\_UNITS label, which can be set in the TDIS Package.  If the time units are not specified, the program still runs, but the table of simulation time does not indicate the time units. An optional LENGTH\_UNITS label can be set in the Discretization Package. Situations in other processes may require that the length or time units be specified.  In such situations, the input instructions will state the requirements. Remember that specifying the unit flags does not enforce consistent use of units.  The user must insure that consistent units are used in all input data.

\subsection{Steady-State Simulations}
A steady-state simulation is represented by a single stress period having a single time step with the storage term set to zero. Setting the number and length of stress periods and time steps is the responsibility of the Timing Module of the \mf framework. The length of the stress period and time step will not affect the head solution because the time derivative is not calculated in a steady-state problem. Setting the storage term to zero is the responsibility of the Storage Package. Most other packages need not "know" that a simulation is steady state.

A GWF Model also can be mixed transient and steady state because each stress period can be designated transient or steady state.  Thus, a GWF Model can start with a steady-state stress period and continue with one or more transient stress periods.  The settings for controlling steady-state and transient options are in the Storage Package.  If the Storage Package is not specified for a GWF Model, then the storage terms are zero and the GWF Model will be steady state.

\subsection{Volumetric Budget}
A summary of all inflows (sources) and outflows (sinks) of water is called a water budget.  The water budget for the GWF Model is termed a volumetric budget because volumes of water and volumetric flow rates are involved; thus strictly speaking, a volumetric budget is not a mass balance, although this term has been used in other model reports.  \mf calculates a water budget for the overall model as a check on the acceptability of the solution, and to provide a summary of the sources and sinks of water to the flow system.  The water budget is printed to the GWF Model Listing File for selected time steps.

Numerical solution techniques for simultaneous equations do not always result in a correct answer; in particular, iterative solvers may stop iterating before a sufficiently close approximation to the solution is attained.  A water budget provides an indication of the overall acceptability of the solution.  The system of equations solved by the model actually consists of a flow continuity statement for each model cell.  Continuity should also exist for the total flows into and out of the model---that is, the difference between total inflow and total outflow should equal the total change in storage.  In the model program, the water budget is calculated independently of the equation solution process, and in this sense may provide independent evidence of a valid solution.

The total budget as printed in the output does not include internal flows between model cells---only flows into or out of the model as a whole. For example, flow to or from rivers, flow to or from constant-head cells, and flow to or from wells are all included in the overall budget terms.  Flow into and out of storage is also considered part of the overall budget inasmuch as accumulation in storage effectively removes water from the flow system and storage release effectively adds water to the flow---even though neither process, in itself, involves the transfer of water into or out of the ground-water regime.  Each hydrologic package calculates its own contribution to the budget.

For every time step, the budget subroutine of each hydrologic package calculates the rate of flow into and out of the system due to the process simulated by the package.  The inflows and outflows for each component of flow are stored separately.  Most packages deal with only one such component of flow.  In addition to flow, the volumes of water entering and leaving the model during the time step are calculated as the product of flow rate and time-step length.  Cumulative volumes, from the beginning of the simulation, are then calculated and stored.

The GWF Model uses the inflows, outflows, and cumulative volumes to write the budget to the Listing File at the times requested by the model user.  When a budget is written, the flow rates for the last time step and cumulative volumes from the beginning of simulation are written for each component of flow.  Inflows are written separately from outflows.  Following the convention indicated above, water entering storage is treated as an outflow (that is, as a loss of water from the flow system) while water released from storage is treated as an inflow (that is, a source of water to the flow system).  In addition, total inflow and total outflow are written, as well as the difference between total inflow and outflow.  The difference is then written as a percentage error, calculated using the formula:

\begin{equation}
D = \frac{100 (IN-OUT)}{(IN + OUT) / 2}
\end{equation}

\noindent where $D$ is the percentage error term, $IN$ is the total inflow to the system, and $OUT$ is the total outflow.

If the model equations are solved correctly, the percentage error should be small.  In general, flow rates may be taken as an indication of solution validity for the time step to which they apply, while cumulative volumes are an indication of validity for the entire simulation up to the time of the output.  The budget is written to the GWF Model Listing File at the end of each stress period whether requested or not.

\subsection{Cell-By-Cell Flows}
In some situations, calculating flow terms for various subregions of the model is useful.  To facilitate such calculations, provision has been made to save flow terms for individual cells in a separate binary file so they can be used in computations external to the model itself.  These individual cell flows are referred to here as ``cell-by-cell'' flow terms and are of four general types: (1) cell-by-cell stress flows, or flows into or from an individual cell caused by one of the external stresses represented in the model, such as evapotranspiration or recharge; (2) cell-by-cell storage terms, which give the rate of accumulation or depletion of storage in an individual cell; and (3) internal cell-by-cell flows, which are actually the flows across individual cell faces---that is, between adjacent model cells.  These four kinds of cell-by-cell flow terms are discussed further in subsequent paragraphs.  To save any of these cell-by-cell terms, two flags in the model input must be set.  The input to the Output Control file indicates the time steps for which cell-by-cell terms are to be saved. In addition, each hydrologic package includes an option called SAVE\_FLOWS that must be set if the cell-by-cell terms computed by that package are to be saved.  Thus, if the appropriate option in the Evapotranspiration Package input is set, cell-by-cell evapotranspiration terms will be saved for each time step for which the saving of cell-by-cell flow is requested through the Output Control Option.  Only flow values are saved in the cell-by-cell files; neither water volumes nor cumulative water volumes are included.  The flow dimensions are volume per unit time, where volume and time are in the same units used for all model input data.  The cell-by-cell flow values are stored in unformatted form to make the most efficient use of disk space; see the Budget File section toward the end of this user guide for information on how the data are written to a file.

The cell-by-cell storage term gives the net flow to or from storage in a variable-head cell.  The net storage for each cell in the grid is saved in transient simulations if the appropriate flags are set.  Withdrawal from storage in the cell is considered positive, whereas accumulation in storage is considered negative.

The cell-by-cell constant-head flow term gives the flow into or out of an individual constant-head cell (specified with the CHD Package).  This term is always associated with the constant-head cell itself, rather than with the surrounding cells that contribute or receive the flow.  A constant-head cell may be surrounded by as many as six adjacent variable-head cells for a regular grid or any number of cells for the other grid types.  The cell-by-cell calculation provides a single flow value for each constant-head cell, representing the algebraic sum of the flows between that cell and all of the adjacent variable-head cells.  A positive value indicates that the net flow is away from the constant-head cell (into the variable-head part of the grid); a negative value indicates that the net flow is into the constant-head cell.

The internal cell-by-cell flow values represent flows across the individual faces of a model cell.  Flows between cells are written in the compressed row storage format, whereby the flow between cell $n$ and each one of its connecting $m$ neighbor cells are contained in a single one-dimensional array.  Flows are positive for the cell in question.  Thus the flow reported for cell $n$ and its connection with cell $m$ is opposite in sign to the flow reported for cell $m$ and its connection with cell $n$.  These internal cell-by-cell flow values are useful in calculations of the groundwater flow into various subregions of the model, or in constructing flow vectors.

Cell-by-cell stress flows are flow rates into or out of the model, at a particular cell, owing to one particular external stress.  For example, the cell-by-cell evapotranspiration term for cell $n$ would give the flow out of the model by evapotranspiration from cell $n$.  Cell-by-cell stress flows are considered positive if flow is into the cell, and negative if out of the cell.

\newpage
\subsection{GWF Model Name File}
The SWF Model Name File specifies the options and packages that are active for a SWF model.  The Name File contains two blocks: OPTIONS  and PACKAGES. The lines in each block can be in any order.  Files listed in the PACKAGES block must exist when the program starts. 

Comment lines are indicated when the first character in a line is one of the valid comment characters.  Commented lines can be located anywhere in the file. Any text characters can follow the comment character. Comment lines have no effect on the simulation; their purpose is to allow users to provide documentation about a particular simulation. 

\vspace{5mm}
\subsubsection{Structure of Blocks}
%\lstinputlisting[style=blockdefinition]{./mf6ivar/tex/swf-nam-options.dat}
%\lstinputlisting[style=blockdefinition]{./mf6ivar/tex/swf-nam-packages.dat}

\vspace{5mm}
\subsubsection{Explanation of Variables}
%\begin{description}
%% DO NOT MODIFY THIS FILE DIRECTLY.  IT IS CREATED BY mf6ivar.py 

\item \textbf{Block: OPTIONS}

\begin{description}
\item \texttt{list}---is name of the listing file to create for this SWF model.  If not specified, then the name of the list file will be the basename of the SWF model name file and the '.lst' extension.  For example, if the SWF name file is called ``my.model.nam'' then the list file will be called ``my.model.lst''.

\item \texttt{PRINT\_INPUT}---keyword to indicate that the list of all model stress package information will be written to the listing file immediately after it is read.

\item \texttt{PRINT\_FLOWS}---keyword to indicate that the list of all model package flow rates will be printed to the listing file for every stress period time step in which ``BUDGET PRINT'' is specified in Output Control.  If there is no Output Control option and ``PRINT\_FLOWS'' is specified, then flow rates are printed for the last time step of each stress period.

\item \texttt{SAVE\_FLOWS}---keyword to indicate that all model package flow terms will be written to the file specified with ``BUDGET FILEOUT'' in Output Control.

\item \texttt{NEWTON}---keyword that activates the Newton-Raphson formulation for surface water flow between connected reaches and stress packages that support calculation of Newton-Raphson terms.

\item \texttt{UNDER\_RELAXATION}---keyword that indicates whether the surface water stage in a reach will be under-relaxed when water levels fall below the bottom of the model below any given cell. By default, Newton-Raphson UNDER\_RELAXATION is not applied.

\end{description}
\item \textbf{Block: PACKAGES}

\begin{description}
\item \texttt{ftype}---is the file type, which must be one of the following character values shown in table~\ref{table:ftype}. Ftype may be entered in any combination of uppercase and lowercase.

\item \texttt{fname}---is the name of the file containing the package input.  The path to the file should be included if the file is not located in the folder where the program was run.

\item \texttt{pname}---is the user-defined name for the package. PNAME is restricted to 16 characters.  No spaces are allowed in PNAME.  PNAME character values are read and stored by the program for stress packages only.  These names may be useful for labeling purposes when multiple stress packages of the same type are located within a single SWF Model.  If PNAME is specified for a stress package, then PNAME will be used in the flow budget table in the listing file; it will also be used for the text entry in the cell-by-cell budget file.  PNAME is case insensitive and is stored in all upper case letters.

\end{description}


%\end{description}

\begin{table}[H]
\caption{Ftype values described in this report.  The \texttt{Pname} column indicates whether or not a package name can be provided in the name file.  The capability to provide a package name also indicates that the SWF Model can have more than one package of that Ftype}
\small
\begin{center}
\begin{tabular*}{\columnwidth}{l l l}
\hline
\hline
Ftype & Input File Description & \texttt{Pname}\\
\hline
DISL6 & Line Network Discretization Input File \\
%IC6 & Initial Conditions Package \\
OC6 & Output Control Option \\
MMR6 & Muskingum Routing Package \\ 
MCT6 & Muskingum-Cunge-Todini Package \\ 
CXS6 & Cross Section Package \\ 
FLW6 & Inflow Package & * \\ 
%OBS6 & Observations Option \\
\hline 
\end{tabular*}
\label{table:ftype}
\end{center}
\normalsize
\end{table}

\vspace{5mm}
\subsubsection{Example Input File}
%\lstinputlisting[style=inputfile]{./mf6ivar/examples/swf-nam-example.dat}



\newpage
\subsection{Structured Discretization (DIS) Input File}
\input{gwf/dis}

\newpage
\subsection{Discretization by Vertices (DISV) Input File}
\input{gwf/disv}

\newpage
\subsection{Unstructured Discretization (DISU) Input File}
\input{gwf/disu}

\newpage
\subsection{Initial Conditions (IC) Package}
\input{gwf/ic}

\newpage
\subsection{Output Control (OC) Option}
Input to the Output Control Option of the Surface Water Flow Model is read from the file that is specified as type ``OC6'' in the Name File. If no ``OC6'' file is specified, default output control is used. The Output Control Option determines how and when outflows are printed to the listing file and/or written to a separate binary output file.  Under the default, outflow and overall flow budget are written to the Listing File at the end of every stress period. The default printout format for concentrations is 10G11.4.  The outflow and overall flow budget are also written to the list file if the simulation terminates prematurely due to failed convergence.

\vspace{5mm}
\subsubsection{Structure of Blocks}
\vspace{5mm}

\noindent \textit{FOR EACH SIMULATION}
\lstinputlisting[style=blockdefinition]{./mf6ivar/tex/swf-oc-options.dat}
\vspace{5mm}
\noindent \textit{FOR ANY STRESS PERIOD}
\lstinputlisting[style=blockdefinition]{./mf6ivar/tex/swf-oc-period.dat}

\vspace{5mm}
\subsubsection{Explanation of Variables}
\begin{description}
% DO NOT MODIFY THIS FILE DIRECTLY.  IT IS CREATED BY mf6ivar.py 

\item \textbf{Block: OPTIONS}

\begin{description}
\item \texttt{BUDGET}---keyword to specify that record corresponds to the budget.

\item \texttt{FILEOUT}---keyword to specify that an output filename is expected next.

\item \texttt{budgetfile}---name of the output file to write budget information.

\item \texttt{BUDGETCSV}---keyword to specify that record corresponds to the budget CSV.

\item \texttt{budgetcsvfile}---name of the comma-separated value (CSV) output file to write budget summary information.  A budget summary record will be written to this file for each time step of the simulation.

\item \texttt{QOUTFLOW}---keyword to specify that record corresponds to qoutflow.

\item \texttt{qoutflowfile}---name of the output file to write conc information.

\item \texttt{STAGE}---keyword to specify that record corresponds to stage.

\item \texttt{stagefile}---name of the output file to write stage information.

\item \texttt{PRINT\_FORMAT}---keyword to specify format for printing to the listing file.

\item \texttt{columns}---number of columns for writing data.

\item \texttt{width}---width for writing each number.

\item \texttt{digits}---number of digits to use for writing a number.

\item \texttt{format}---write format can be EXPONENTIAL, FIXED, GENERAL, or SCIENTIFIC.

\end{description}
\item \textbf{Block: PERIOD}

\begin{description}
\item \texttt{iper}---integer value specifying the starting stress period number for which the data specified in the PERIOD block apply.  IPER must be less than or equal to NPER in the TDIS Package and greater than zero.  The IPER value assigned to a stress period block must be greater than the IPER value assigned for the previous PERIOD block.  The information specified in the PERIOD block will continue to apply for all subsequent stress periods, unless the program encounters another PERIOD block.

\item \texttt{SAVE}---keyword to indicate that information will be saved this stress period.

\item \texttt{PRINT}---keyword to indicate that information will be printed this stress period.

\item \texttt{rtype}---type of information to save or print.  Can be BUDGET.

\item \texttt{ocsetting}---specifies the steps for which the data will be saved.

\begin{lstlisting}[style=blockdefinition]
ALL
FIRST
LAST
FREQUENCY <frequency>
STEPS <steps(<nstp)>
\end{lstlisting}

\item \texttt{ALL}---keyword to indicate save for all time steps in period.

\item \texttt{FIRST}---keyword to indicate save for first step in period. This keyword may be used in conjunction with other keywords to print or save results for multiple time steps.

\item \texttt{LAST}---keyword to indicate save for last step in period. This keyword may be used in conjunction with other keywords to print or save results for multiple time steps.

\item \texttt{frequency}---save at the specified time step frequency. This keyword may be used in conjunction with other keywords to print or save results for multiple time steps.

\item \texttt{steps}---save for each step specified in STEPS. This keyword may be used in conjunction with other keywords to print or save results for multiple time steps.

\end{description}


\end{description}

\vspace{5mm}
\subsubsection{Example Input File}
\lstinputlisting[style=inputfile]{./mf6ivar/examples/swf-oc-example.dat}


\newpage
\subsection{Observation (OBS) Utility for a GWF Model}
\input{gwf/gwf-obs}

\newpage
\subsection{Node Property Flow (NPF) Package}
\input{gwf/npf}

\newpage
\subsection{Time-Varying Hydraulic Conductivity (TVK) Package}
\input{gwf/tvk}

\newpage
\subsection{Horizontal Flow Barrier (HFB) Package}
\input{gwf/hfb}

\newpage
\subsection{Storage (STO) Package}
\input{gwf/sto}

\newpage
\subsection{Time-Varying Storage (TVS) Package}
\input{gwf/tvs}

\newpage
\subsection{Skeletal Storage, Compaction, and Subsidence (CSUB) Package}
\input{gwf/csub}

\newpage
\subsection{Buoyancy (BUY) Package}
\input{gwf/buy}

\IfFileExists{develop.version}{
\newpage
\subsection{Sea Water Intrusion (SWI) Package}
Input to the Sea Water Intrusion (SWI) Package is read from the file that has type ``SWI6'' in the Name File.  If the SWI Package is included for a model, then the model will calculate the position of the freshwater interface in each cell and calculate flow accordingly.

For some applications, it may be necessary to specify the saltwater head as an array that changes with time.  As described in the options block for TVA6, the  \hyperref[sec:tva]{Time Variable Array (TVA) Utility} can be used to specify the saltwater head .

\vspace{5mm}
\subsubsection{Structure of Blocks}

\vspace{5mm}
\noindent \textit{FOR EACH SIMULATION}
\lstinputlisting[style=blockdefinition]{./mf6ivar/tex/gwf-swi-options.dat}
\lstinputlisting[style=blockdefinition]{./mf6ivar/tex/gwf-swi-griddata.dat}
%\vspace{5mm}
%\noindent \textit{FOR ANY STRESS PERIOD}
%\lstinputlisting[style=blockdefinition]{./mf6ivar/tex/gwf-buy-period.dat}

\vspace{5mm}
\subsubsection{Explanation of Variables}
\begin{description}
% DO NOT MODIFY THIS FILE DIRECTLY.  IT IS CREATED BY mf6ivar.py 

\item \textbf{Block: OPTIONS}

\begin{description}
\item \texttt{SALTWATER}---keyword to indicate that this model is a saltwater model

\item \texttt{ZETA}---keyword to specify that record corresponds to zeta.

\item \texttt{FILEOUT}---keyword to specify that an output filename is expected next.

\item \texttt{zetafile}---name of the binary output file to write zeta information.

\end{description}
\item \textbf{Block: GRIDDATA}

\begin{description}
\item \texttt{zetastrt}---is the initial (starting) zeta surface---that is, interface elevation at the beginning of the GWF Model simulation.

\end{description}


\end{description}

\vspace{5mm}
\subsubsection{Example Input File}
\lstinputlisting[style=inputfile]{./mf6ivar/examples/gwf-swi-example.dat}


}{}

\newpage
\subsection{Viscosity (VSC) Package}
\input{gwf/vsc}

\newpage
\subsection{Constant-Head (CHD) Package}
\input{gwf/chd}

\newpage
\subsection{Well (WEL) Package}
\input{gwf/wel}

\newpage
\subsection{Drain (DRN) Package}
\input{gwf/drn}

\newpage
\subsection{River (RIV) Package}
\input{gwf/riv}

\newpage
\subsection{General-Head Boundary (GHB) Package}
\input{gwf/ghb}

\newpage
\subsection{Recharge (RCH) Package -- List-Based Input}
\input{gwf/rch}

\newpage
\subsection{Recharge (RCH) Package -- Array-Based Input}
\input{gwf/rcha}

\newpage
\subsection{Evapotranspiration (EVT) Package -- List-Based Input}
\input{gwf/evt}

\newpage
\subsection{Evapotranspiration (EVT) Package -- Array-Based Input}
\input{gwf/evta}

\newpage
\subsection{Multi-Aquifer Well (MAW) Package}
\input{gwf/maw}

\newpage
\subsection{Streamflow Routing (SFR) Package}
\input{gwf/sfr}

\newpage
\subsection{Lake (LAK) Package}
\input{gwf/lak}

\newpage
\subsection{Unsaturated Zone Flow (UZF) Package}
\input{gwf/uzf}

\newpage
\subsection{Water Mover (MVR) Package}
\input{gwf/mvr}

\newpage
\subsection{Ghost-Node Correction (GNC) Package}
\input{gwf/gnc}

\newpage
\subsection{Groundwater Flow (GWF) Exchange}
\input{gwf/gwf-gwf}

